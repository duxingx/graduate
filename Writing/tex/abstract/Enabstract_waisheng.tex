%\newpage                                    %
%\thispagestyle{plain}                       %
%\addcontentsline{toc}{chapter}{ABSTRACT}
\begin{center}
{\zihao{2}\enbiaoti}
\vspace{1cm}

%{\zihao{3}

% \begin{tabular}{c}
%   \englishname \\
%   % Major:      & \researcharea   \\
%   Directed by \advisor \\
% \end{tabular}
%}
\vspace{1.5cm}

{\huge\bf ABSTRACT}
\end{center}
\vspace{.6cm}

% ������Ӣ��ժҪ��


It is a significant research work to accurately predict the mortality of patients with severe 
renal failure. It can make early assessment of patients' disease and preliminary prediction 
of clinical outcomes, allowing clinicians to carry out early intervention in severe patients 
according to the expected outcome and reasonable care for patients to avoid their disease 
progression. In addition, it also plays an important role in reducing the risk of death 
and allocating medical resources reasonably. Survival analysis is a statistical method 
that studies the timing of events, including death, disease recurrence, customer loss, 
and so on. It has a wide range of applications in biomedical, engineering, and finance. 
At present, many survival analysis models have been developed, but the comparison of these 
survival analysis models in the data of patients with severe renal failure has not been 
well applied. At present, the evaluation method of survival analysis results is relatively 
simple, and few studies have carried out comprehensive evaluation.


In view of the above problems, this paper has carried on the in-depth research. Firstly, 
by comprehensively considering missing values and outliers in the data of patients with 
severe renal failure, different processing methods were used for variables of different 
types, different missing rates and different importance, so as to reduce the influence 
of data quality on experimental results. Then, six survival statistical models, including 
COX-ElasticNet, RSF, DeepSurv, Cox-CC, Cox-Time and LogisitcHard, were applied to the 
clinical data of patients with severe renal failure to predict and analyze the mortality 
of patients. Different survival analysis statistical models were compared and evaluated 
in all aspects by consistency index, $C^{td}$,  BS, IBS  and ROC-Time. For the proportional 
risk model, this paper uses consistency index to evaluate the differentiation of the model, 
and for the non-proportional risk model, this paper uses $C^{td}$index to evaluate. ROC-Time 
was used to evaluate the prediction accuracy and stability of the model, and BS and IBS were 
used to evaluate the calibration degree of the model. Finally, the main risk factors and 
protective factors of renal failure patients were discussed.


The results of this paper show that, for the six models studied, the Cox-CC model is the best 
model, and the Cox-Time model is the second. Cox-ElasticNet and RSF model have the best 
performance in terms of prediction accuracy, Cox-CC and Cox-Time model have the best 
performance in terms of model stability, and RSF model has the best performance in terms 
of model calibration degree. Cox-CC and DeepSurv performed best in terms of long-term 
prediction accuracy and stability. For the short to medium term, RSF model is the best 
performer. Finally, the Cox-ElasticNet model was used to calculate the risk factors 
and protective factors of patients with severe renal failure. The most important risk 
factors were: minimum blood urea nitrogen within 24 hours after admission to ICU; The 
most important protective factor was the minimum oxygenation index within 24 hours of 
admission to the ICU. The establishment of such an optimal survival analysis statistical 
model is expected to provide necessary reference for clinical medical staff to develop 
a more scientific and reasonable prognostic diagnosis method, so as to play a positive 
impact on the rehabilitation of patients with severe renal failure.


\vspace{0.2cm}

\noindent{\zihao{-4}{\bf Key words:} Deep learning;\; Renal failure;\; ICU}
