%\newpage                                    %
%\thispagestyle{plain}                       %
%\addcontentsline{toc}{chapter}{ABSTRACT}
\begin{center}
{\zihao{2}\enbiaoti}
\vspace{3cm}
%
%%{\zihao{3}
%
%\begin{tabular}{ll}
%  Name:       & \englishname \\
%  Major:      & \researcharea   \\
%  Supervisor: & \advisor \\
%\end{tabular}
%%}
%\vspace{1.5cm}

{\huge\bf ABSTRACT}
\end{center}
\vspace{.6cm}

% ������Ӣ��ժҪ��
Objectives��To evaluate the predictive effectiveness of different machine learning algorithms in predicting opioid use. To study the influence of social and economic factors on opioid use and explore the causes of opioid addiction.

Methods��From 2010 to 2016, Kentucky, West Virginia, Virginia, Ohio and Pennsylvania opioids for descriptive analysis, to extract the social, economic attribute, by K Nearest Neighbor, Decision Tree, and Random Forest, Support Vector Machine, Artificial Neural Network and Logistics 6 kinds of machine learning algorithms to establish regression model, evaluation and comparing the pros and cons of different algorithm and prediction.

Results��There were 56 kinds of opioid drugs reported, and the distribution of various opioids was mainly concentrated in heroin, hydrocodone and hydrocodone, accounting for 53.24\%, 19.79\% and 8.57\% respectively. Accuracy of Support Vector Machine model $> 0.8$, and the accuracy of Artificial Neural Network model $> 0.7$.

Conclusion��The performance of Support Vector Machine model and Artificial Neural Network model in predicting effect is ideal. Compared with the traditional prediction models, Support Vector Machines and Neural Network models can easily adjust the parameters and generate better prediction models.

\vspace{0.5cm}

\noindent{\zihao{-4}{\bf Key words:} Opioids;\; Visualization;\; Machine learning;\; Prediction model}
